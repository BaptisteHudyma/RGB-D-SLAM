My take on a SLAM, based on shape primitive recognition with a RGB-\/D camera. The system uses depth informations as a connected 2D graph to extract primitive shapes, and use them with 3D points and lines to estimate it\textquotesingle{}s position in space. A map is created with those points, lines and primitive shapes.

The primitive detection is based on \href{https://arxiv.org/pdf/1803.02380.pdf}{\texttt{ Fast Cylinder and Plane Extraction from Depth Cameras for Visua Odometry}}. The pose estimation is based on \href{https://www.ncbi.nlm.nih.gov/pmc/articles/PMC6165120/}{\texttt{ Lightweight Visual Odometry for Autonomous Mobile Robots}}.

packages 
\begin{DoxyCode}{0}
\DoxyCodeLine{opencv}
\DoxyCodeLine{Eigen}
\DoxyCodeLine{g2o}

\end{DoxyCode}


How to build Build 
\begin{DoxyCode}{0}
\DoxyCodeLine{mkdir build \&\& cd build}
\DoxyCodeLine{cmake ..}
\DoxyCodeLine{make}

\end{DoxyCode}


How to use 
\begin{DoxyCode}{0}
\DoxyCodeLine{./rgbdslam}

\end{DoxyCode}
 Parameters 
\begin{DoxyCode}{0}
\DoxyCodeLine{-\/h Display the help}
\DoxyCodeLine{-\/f path to the file containing the data (depth, rgb, cam parameters)}
\DoxyCodeLine{-\/c use cylinder detection }
\DoxyCodeLine{-\/i index of starting frame (> 0)}

\end{DoxyCode}


Check mem errors 
\begin{DoxyCode}{0}
\DoxyCodeLine{valgrind -\/-\/suppressions=/usr/share/opencv4/valgrind.supp -\/-\/suppressions=/usr/share/opencv4/valgrind\_3rdparty.supp ./rgbdslam}

\end{DoxyCode}
 